% ================================================================
% Hauptdatei – APA7-Vorlage für wissenschaftliche Arbeiten
% ================================================================
% ================================================================
% APA7-konforme Dokumenteinstellungen
% Geeignet für akademische Arbeiten an Hochschulen
% ================================================================

% Dokumentenklasse
% openany: kein Leerseiten-Umbruch vor jedem Kapitel
% chapterprefix=false: kein "Kapitel X" vor dem Titel
\documentclass[
  12pt,
  a4paper,
  oneside,
  openany,
  numbers=noendperiod,
  chapterprefix=false
]{scrbook}


% ---- Seitenränder nach APA7: 2,54 cm (1 Zoll) auf allen Seiten ----
\usepackage[
  top=25.4mm,
  bottom=25.4mm,
  left=25.4mm,
  right=25.4mm
]{geometry}


% ---- Tweaks für KOMA-Script ----
\usepackage{scrhack}


% ---- Schriftart: Times New Roman, 12 pt (APA7-konform) ----
% newtxtext/newtxmath: moderner Ersatz für das veraltete mathptmx,
% liefert hochwertigeres Times New Roman für pdflatex
\usepackage[T1]{fontenc}
\usepackage[utf8]{inputenc}
\usepackage{newtxtext}   % Times New Roman für Fließtext
\usepackage{newtxmath}   % Passende Mathematikschrift
% Alternative bei Verwendung von LuaLaTeX oder XeLaTeX:
%   \usepackage{fontspec}
%   \setmainfont{Times New Roman}


% ---- Sprache: Deutsch ----
\usepackage[ngerman]{babel}


% ---- Mikrotypografie (verbesserte Buchstabenabstände) ----
\usepackage[stretch=10]{microtype}


% ---- Doppelter Zeilenabstand (APA7-Pflicht) ----
\usepackage[doublespacing]{setspace}


% ---- Absatzeinzug nach APA7: 1,27 cm (0,5 Zoll) ----
% Kein zusätzlicher Abstand zwischen Absätzen
\setlength{\parindent}{1.27cm}
\setlength{\parskip}{0pt}


% ---- Captions (Abbildungs- und Tabellenbeschriftungen) ----
\usepackage{caption}


% ---- Grafiken ----
\usepackage{graphicx}


% ---- Positionierung von Abbildungen und Tabellen ----
\usepackage{float}


% ---- Tabellen ----
\usepackage{tabularx}                 % Tabellen mit flexiblen Spaltenbreiten
\usepackage{longtable}                % Seitenübergreifende Tabellen
\usepackage{booktabs}                 % APA7-Tabellenlinien (keine vertikalen Linien)


% ---- Farben ----
\usepackage[table]{xcolor}


% ---- Hyperlinks (ohne farbige Rahmen) ----
\usepackage[hidelinks]{hyperref}


% ---- Literaturverwaltung nach APA7 ----
% Benötigt: biblatex-apa (CTAN-Paket)
\usepackage{csquotes}
\usepackage[
  style=apa,
  backend=biber
]{biblatex}
\DeclareLanguageMapping{ngerman}{ngerman-apa}
\addbibresource{literatur/quellen.bib}


% ---- Kopf- und Fußzeile ----
\usepackage[automark]{scrlayer-scrpage}  % headsepline entfernt: APA7 hat keine Trennlinie
% ================================================================
% Kopf- und Fußzeile
%   – Links:  Aktueller Kapitelname (wechselt automatisch)
%   – Rechts: Seitenzahl
%
% Hinweis APA7:
%   Streng nach APA7 (Manuskript zur Publikation) wäre links ein
%   fester Kurztitel (Running Head). Für Abschlussarbeiten und
%   Hausarbeiten an deutschen Hochschulen ist der Kapitelname
%   jedoch üblicher und prüferseitig erwartet.
%
%   Für festen Kurztitel: \ihead{\small\MakeUppercase{\runninghead}}
%   und \renewcommand{\runninghead}{KURZTITEL} in main.tex setzen.
% ================================================================

\automark{chapter}
\ihead{\small\leftmark}   % Links: aktueller Kapitelname
\chead{}                   % Mitte: leer
\ohead{\thepage}           % Rechts: Seitenzahl

\ifoot*{}
\cfoot*{}
\ofoot*{}

\pagestyle{scrheadings}


% ================================================================
% APA7-Formatierungsanpassungen
% ================================================================


% ---- Serifenschrift für alle Überschriften beibehalten ----
\addtokomafont{disposition}{\rmfamily}


% ================================================================
% APA7 – Keine Nummerierung der Überschriften
% ================================================================
\setcounter{secnumdepth}{-1}


% ================================================================
% APA7 – Fünf Überschriftenebenen
% ================================================================

% Ebene 1 – chapter: Zentriert, Fett, Titelschreibung
% Beispiel:  Einleitung
\setkomafont{chapter}{\normalsize\bfseries}
\renewcommand*{\raggedchapter}{\centering}
\RedeclareSectionCommand[
  beforeskip=0pt,
  afterskip=\baselineskip
]{chapter}

% Ebene 2 – section: Linksbündig, Fett, Titelschreibung
% Beispiel:  Theoretischer Hintergrund
% afterskip muss > 0 sein, sonst behandelt KOMA-Script die Überschrift als Runin
\setkomafont{section}{\normalsize\bfseries}
\RedeclareSectionCommand[
  beforeskip=\baselineskip,
  afterskip=1sp
]{section}

% Ebene 3 – subsection: Linksbündig, Fett und Kursiv, Titelschreibung
% Beispiel:  Theoretische Grundlagen
\setkomafont{subsection}{\normalsize\bfseries\itshape}
\RedeclareSectionCommand[
  beforeskip=\baselineskip,
  afterskip=1sp
]{subsection}

% Ebene 4 – subsubsection: Eingerückt, Fett, Titelschreibung, Punkt, Fließtext folgt
% Beispiel:  Kommunikation im Team.  Text beginnt hier …
\setkomafont{subsubsection}{\normalsize\bfseries}
\RedeclareSectionCommand[
  style=runin,
  beforeskip=0pt,
  afterskip=1em,
  indent=\parindent
]{subsubsection}

% Ebene 5 – paragraph: Eingerückt, Fett und Kursiv, Titelschreibung, Punkt, Fließtext folgt
% Beispiel:  Verbale Kommunikation.  Text beginnt hier …
\setkomafont{paragraph}{\normalsize\bfseries\itshape}
\RedeclareSectionCommand[
  style=runin,
  beforeskip=0pt,
  afterskip=1em,
  indent=\parindent
]{paragraph}


% ================================================================
% APA7 – Tabellen
% APA7-Tabellen haben:
%   • Titel über der Tabelle (Nummer fett, Titel kursiv, neue Zeile)
%   • Keine vertikalen Linien
%   • Horizontale Linien: nur über und unter dem Kopf sowie am Ende
%   • Anmerkungen unter der Tabelle (kursiv „Anmerkung.")
% Empfehlung: Tabellen mit \toprule, \midrule, \bottomrule (booktabs) erstellen
% ================================================================
\setlength{\arrayrulewidth}{0.5pt}


% ================================================================
% APA7 – Blockzitat (quote-Umgebung)
% APA7 §8.27: Blockzitate werden nur linksseitig eingerückt (1,27 cm),
% nicht beidseitig wie LaTeXs Standard-quote.
% ================================================================
\renewenvironment{quote}{%
  \list{}{%
    \leftmargin=1.27cm
    \rightmargin=0pt
    \topsep=0pt
    \parsep=0pt
    \itemsep=0pt
  }%
  \item\relax
}{%
  \endlist
}


% ================================================================
% APA7 – Caption-Format für Abbildungen und Tabellen
% Aufbau: „Abbildung 1" (fett) auf eigener Zeile, Titel (kursiv) darunter
% \DeclareCaptionFormat ist robuster als labelsep=newline in KOMA-Script
% ================================================================
% labelfont=bf macht das Label fett (robuster als \textbf im Format)
% textfont=it macht den Beschreibungstext kursiv
% Das Format {#1#2\\#3} erzeugt: Label + Trenner → Zeilenumbruch → Text
\DeclareCaptionFormat{apa7}{#1#2\\#3}

\captionsetup[figure]{
  format=apa7,
  labelfont=bf,
  textfont=it,
  labelsep=space,
  justification=raggedright,
  singlelinecheck=false
}
\captionsetup[table]{
  format=apa7,
  labelfont=bf,
  textfont=it,
  labelsep=space,
  justification=raggedright,
  singlelinecheck=false,
  position=above
}



% ================================================================
% Titelseite – gewünschtes Layout einkommentieren
% ================================================================

% Option A: Abschlussarbeit (Bachelor/Master/Diplom)
% ================================================================
% Titelseite: Abschlussarbeit (APA7-Studentenpapier-Format)
% Verwendete Befehle in main.tex:
%   \title{...}            Titel der Arbeit
%   \author{...}           Verfasser/in
%   \matrikelnr{...}       Matrikelnummer
%   \submitDate{...}       Abgabedatum
%   \firstExaminer{...}    Erstgutachter/in
%   \secondExaminer{...}   Zweitgutachter/in
%   \gradeType{...}        Angestrebter Abschluss
%   \institution{...}      Hochschule / Fachbereich
% ================================================================

\makeatletter

\newcommand*{\gradeType}[1]{\gdef\@gradeType{#1}}
\newcommand*{\firstExaminer}[1]{\gdef\@firstExaminer{#1}}
\newcommand*{\secondExaminer}[1]{\gdef\@secondExaminer{#1}}
\newcommand*{\matrikelnr}[1]{\gdef\@matrikelnr{#1}}
\newcommand*{\submitDate}[1]{\gdef\@submitDate{#1}}
\newcommand*{\institution}[1]{\gdef\@institution{#1}}

% Standardwerte (werden durch main.tex überschrieben)
\gradeType{Bachelor of Science (B.Sc.)}
\firstExaminer{}
\secondExaminer{}
\matrikelnr{}
\submitDate{\today}
\institution{}

\renewcommand*{\maketitle}{%
  \begin{titlepage}%
    \newgeometry{
      top=25.4mm,
      bottom=25.4mm,
      left=25.4mm,
      right=25.4mm
    }%
    \begin{center}%
      \setstretch{2.0}%

      % Institution
      \vspace*{\fill}
      {\normalsize \@institution \par}%

      \vspace{3\baselineskip}

      % Titel (fett, APA7)
      {\normalsize \bfseries \@title \par}%

      \vspace{2\baselineskip}

      % Verfasser/in
      {\normalsize \@author \par}%
      {\normalsize Matrikelnummer: \@matrikelnr \par}%

      \vspace{2\baselineskip}

      % Arbeit und Abschluss
      {\normalsize Abschlussarbeit zur Erlangung des akademischen Grades \par}%
      {\normalsize \@gradeType \par}%

      \vspace{2\baselineskip}

      % Prüfer/innen
      {\normalsize Erstgutachter/in: \@firstExaminer \par}%
      {\normalsize Zweitgutachter/in: \@secondExaminer \par}%

      \vspace{2\baselineskip}

      % Datum
      {\normalsize Abgabedatum: \@submitDate \par}%

      \vspace*{\fill}
    \end{center}%
    \restoregeometry%
  \end{titlepage}%
}

\makeatother

\gradeType{Bachelor of Science (B.Sc.)}
\secondExaminer{Vorname Nachname}

% Option B: Hausarbeit / Seminararbeit
% % ================================================================
% Titelseite: Hausarbeit / Seminararbeit (APA7-Format)
% Verwendete Befehle in main.tex:
%   \title{...}            Titel der Hausarbeit
%   \author{...}           Verfasser/in
%   \matrikelnr{...}       Matrikelnummer
%   \submitDate{...}       Abgabedatum
%   \firstExaminer{...}    Dozent/in
%   \modulName{...}        Modulbezeichnung
%   \institution{...}      Hochschule / Fachbereich
% ================================================================

\makeatletter

% Befehle (sofern nicht bereits in graduation.tex definiert)
\providecommand*{\firstExaminer}[1]{\gdef\@firstExaminer{#1}}
\providecommand*{\matrikelnr}[1]{\gdef\@matrikelnr{#1}}
\providecommand*{\submitDate}[1]{\gdef\@submitDate{#1}}
\providecommand*{\institution}[1]{\gdef\@institution{#1}}
\newcommand*{\modulName}[1]{\gdef\@modulName{#1}}

% Standardwerte
\firstExaminer{}
\matrikelnr{}
\submitDate{\today}
\institution{}
\modulName{}

\renewcommand*{\maketitle}{%
  \begin{titlepage}%
    \newgeometry{
      top=25.4mm,
      bottom=25.4mm,
      left=25.4mm,
      right=25.4mm
    }%
    \begin{center}%
      \setstretch{2.0}%

      % Institution
      \vspace*{\fill}
      {\normalsize \@institution \par}%

      \vspace{3\baselineskip}

      % Titel (fett, APA7)
      {\normalsize \bfseries \@title \par}%

      \vspace{2\baselineskip}

      % Verfasser/in
      {\normalsize \@author \par}%
      {\normalsize Matrikelnummer: \@matrikelnr \par}%

      \vspace{2\baselineskip}

      % Modul und Dozent/in
      {\normalsize \@modulName \par}%
      {\normalsize Dozent/in: \@firstExaminer \par}%

      \vspace{2\baselineskip}

      % Datum
      {\normalsize Abgabedatum: \@submitDate \par}%

      \vspace*{\fill}
    \end{center}%
    \restoregeometry%
  \end{titlepage}%
}

\makeatother

% \modulName{Modul: Modulname}


% ================================================================
% Angaben zur Arbeit
% ================================================================
\title{Titel der Arbeit}
\author{Vorname Nachname}
\matrikelnr{0000000}
\submitDate{\today}
\firstExaminer{Vorname Nachname}
\institution{Name der Hochschule \\ Fachbereich}

% Running Head: Kurztitel in Großbuchstaben, max. 50 Zeichen
% Erscheint oben links auf jeder Seite (APA7-Manuskriptformat)
% Für reine Studentenarbeiten: \ihead{} in einstellungen/kopfzeile.tex auskommentieren
\renewcommand{\runninghead}{KURZTITEL DER ARBEIT}


% ================================================================
% Dokumentbeginn
% ================================================================
\begin{document}

\pagenumbering{alph}
\maketitle

\pagenumbering{Roman}

% Abstract (eigene Seite, 150–250 Wörter)
% ================================================================
% Abstract (APA7)
% – Eigene Seite, direkt nach dem Titelblatt
% – Überschrift „Abstract" zentriert, fett (Ebene 1)
% – Kein Einzug im ersten Absatz des Abstracts
% – Umfang: 150–250 Wörter
% – Am Ende optional: Schlüsselwörter (kursiv „Schlüsselwörter:")
% ================================================================

\chapter*{Abstract}
\addcontentsline{toc}{chapter}{Abstract}

% Kein Einzug für den ersten Absatz des Abstracts
\noindent
Hier den Abstract der Arbeit einfügen (150–250 Wörter). Der Abstract fasst Fragestellung, Methode, zentrale Ergebnisse und Schlussfolgerung der Arbeit knapp zusammen. Er steht auf einer eigenen Seite und wird nicht eingerückt.

\noindent\textit{Schlüsselwörter:} Schlüsselwort 1, Schlüsselwort 2, Schlüsselwort 3

\clearpage

% Inhaltsverzeichnis
\tableofcontents
\clearpage

% Abbildungsverzeichnis (einkommentieren, wenn Abbildungen vorhanden)
% \listoffigures \clearpage

% Tabellenverzeichnis (einkommentieren, wenn Tabellen vorhanden)
% \listoftables \clearpage


% ================================================================
% Hauptteil – Grundgerüst einer wissenschaftlichen Arbeit
% ================================================================
\pagenumbering{arabic}

% 1. Einleitung
% ================================================================
% Einleitung
% Überschrift Ebene 1: Zentriert, Fett (APA7)
% ================================================================

\chapter{Einleitung}

Hier den Text der Einleitung einfügen. Die Einleitung führt in das Thema ein, benennt die Problemstellung und formuliert die Forschungsfrage bzw.\ die Zielsetzung der Arbeit. Sie gibt außerdem einen Überblick über den Aufbau der Arbeit.


% ================================================================
% Hinweise zu APA7-Zitierweise (zum Entfernen nach Fertigstellung)
% ================================================================
%
% Kurzzitat im Text:
%   (Nachname, Jahr, S. XX)             → z.B. (Müller, 2021, S. 45)
%   Nachname (Jahr)                     → z.B. Müller (2021)
%   Zwei Autor/innen: (Müller \& Schneider, 2021)
%   Drei und mehr: (Müller et al., 2021)
%
% Indirektes Zitat (Paraphrase):
%   Laut Müller (2021) zeigt sich, dass …
%   (Müller, 2021)
%
% Wörtliches Zitat bis 39 Wörter – im Fließtext in Anführungszeichen:
%   „Wortlaut des Zitats" (Müller, 2021, S. 45)
%
% Wörtliches Zitat ab 40 Wörter – als Blockzitat (eingerückt, ohne Anführungszeichen):
%   \begin{quote}
%     Wortlaut des langen Zitats. (Müller, 2021, S. 45)
%   \end{quote}
%
% Literatureintrag in literatur/quellen.bib einfügen,
% dann im Text mit \autocite{Schlüssel} zitieren.
%
% ================================================================
 \clearpage

% 2. Theoretischer Hintergrund / Literaturübersicht
% ================================================================
% Kapitel 2: Theoretischer Hintergrund
% ================================================================

\chapter{Theoretischer Hintergrund}

Einführung in den theoretischen Rahmen der Arbeit. Zentrale Konzepte, Modelle und Theorien werden vorgestellt und zueinander in Bezug gesetzt. Der Abschnitt schließt mit der Herleitung der Forschungsfrage.


\section{Begriffe und Definitionen}

Klärung zentraler Begriffe, die für die Arbeit relevant sind. Definitionen werden aus der wissenschaftlichen Literatur abgeleitet und begründet \autocite{Buch2021}.


\section{Forschungsstand}

Überblick über den aktuellen Stand der Forschung zum Thema. Zentrale Studien und Befunde werden systematisch dargestellt und bewertet. Forschungslücken werden identifiziert und begründet.

\subsection{Bisherige Befunde}

Zusammenfassung und kritische Einordnung der relevanten Studien. Übereinstimmungen und Widersprüche in der Literatur werden herausgearbeitet.

\subsection{Forschungslücken}

Benennung offener Fragen, die in der bisherigen Forschung nicht oder unzureichend beantwortet wurden und die Grundlage für die vorliegende Arbeit bilden.


\section{Forschungsfrage und Zielsetzung}

Formulierung der zentralen Forschungsfrage auf Basis der identifizierten Forschungslücke. Die Zielsetzung der Arbeit wird präzise benannt und in den wissenschaftlichen Kontext eingeordnet.
 \clearpage

% 3. Methodik
% ================================================================
% Kapitel 3: Methodik
% ================================================================

\chapter{Methodik}

Überblick über das methodische Vorgehen der Arbeit. Die Wahl der Methode wird begründet und in Bezug auf die Forschungsfrage gerechtfertigt. Das Kapitel folgt dem IMRAD-Schema (Introduction, Methods, Results, and Discussion), das in empirischen wissenschaftlichen Arbeiten weit verbreitet ist.


\section{Forschungsdesign}

Beschreibung und Begründung des gewählten Forschungsdesigns (z.B.\ systematische Literaturrecherche, qualitative Inhaltsanalyse, quantitative Querschnittsstudie). Das Design wird in Bezug auf die Forschungsfrage begründet \autocite{Kapitel2020}.

\subsection{Forschungsparadigma}

Einordnung der Arbeit in das qualitative bzw.\ quantitative Forschungsparadigma und Begründung der Wahl.

\subsection{Gütekriterien}

Darstellung der angewandten Gütekriterien (z.B.\ Reliabilität, Validität, Transferierbarkeit, Glaubwürdigkeit) und Maßnahmen zu ihrer Sicherstellung.


\section{Datenerhebung}

Beschreibung des Vorgehens bei der Datenerhebung. Bei Literaturarbeiten: Suchstrategie, Datenbanken, Ein- und Ausschlusskriterien, PRISMA-Flow. Bei empirischen Arbeiten: Erhebungsinstrument, Feldzugang, Sampling.

\subsection{Suchstrategie und Datenbankrecherche}

Die Literaturrecherche wurde in den Datenbanken \ldots\ durchgeführt. Folgende Suchbegriffe wurden verwendet: \ldots\ Die Suche ergab insgesamt \ldots\ Treffer.

\subsection{Ein- und Ausschlusskriterien}

\begin{table}[H]
  \caption{Ein- und Ausschlusskriterien der Literaturrecherche}
  \label{tab:kriterien}
  \begin{tabularx}{\textwidth}{@{}lXX@{}}
    \toprule
    \textbf{Kriterium} & \textbf{Einschluss} & \textbf{Ausschluss} \\
    \midrule
    Sprache          & Deutsch, Englisch                        & Alle anderen Sprachen        \\
    Erscheinungsjahr & Ab 2014                                  & Vor 2014                     \\
    Studiendesign    & RCT, Kohortenstudie, qualitative Studie  & Fallberichte, Expertenmeinungen \\
    Thema            & Direkt relevant für Forschungsfrage      & Randthemen                   \\
    \bottomrule
  \end{tabularx}
  \par\vspace{2pt}
  {\raggedright\footnotesize\textit{Anmerkung.} Eigene Darstellung.\par}
\end{table}


\section{Datenauswertung}

Beschreibung der Auswertungsmethode (z.B.\ qualitative Inhaltsanalyse nach Mayring, thematische Analyse nach Braun \& Clarke, statistische Auswertung mit SPSS). Vorgehen wird schrittweise erläutert.

\subsection{Auswertungsverfahren}

Begründung der Wahl des Auswertungsverfahrens und Beschreibung der einzelnen Analyseschritte.

\subsection{Kategorienbildung}

Bei qualitativen Arbeiten: Beschreibung der Kategorienbildung (deduktiv, induktiv oder kombiniert) und der Kodierregeln.


\section{Ethische Überlegungen}

Reflexion ethischer Aspekte der Studie (z.B.\ Datenschutz, informierte Einwilligung, Anonymisierung, Genehmigung durch Ethikkommission). Bei Literaturarbeiten: Reflexion möglicher Verzerrungen (Bias).
 \clearpage

% 4. Ergebnisse
% ================================================================
% Kapitel 4: Ergebnisse
% ================================================================

\chapter{Ergebnisse}

Darstellung der Ergebnisse der Untersuchung. Die Ergebnisse werden neutral und wertfrei präsentiert – ohne Interpretation oder Diskussion, diese erfolgt im nachfolgenden Kapitel. Die Gliederung folgt der Struktur der Forschungsfrage bzw.\ der Kategorien der Auswertung \autocite{Muster2023}.


\section{Ergebnis zu Teilaspekt 1}

Beschreibung der Befunde zu Teilaspekt 1. Ergebnisse werden durch Tabellen und Abbildungen veranschaulicht, die im Text erklärt und interpretationsfrei beschrieben werden.


\section{Ergebnis zu Teilaspekt 2}

Beschreibung der Befunde zu Teilaspekt 2.


\section{Zusammenfassung der Ergebnisse}

Überblicksartige Zusammenfassung aller zentralen Ergebnisse im Hinblick auf die Forschungsfrage. Diese Zusammenfassung bereitet die nachfolgende Diskussion vor.
 \clearpage

% 5. Diskussion
% ================================================================
% Kapitel 5: Diskussion
% ================================================================

\chapter{Diskussion}

Interpretation und kritische Einordnung der Ergebnisse in den theoretischen Rahmen und den Forschungsstand. Die Diskussion beantwortet die Forschungsfrage, setzt die Befunde in Bezug zur Literatur und reflektiert die Stärken und Grenzen der Arbeit \autocite{Schaefer2022}.


\section{Interpretation der Ergebnisse}

Interpretation der zentralen Befunde im Licht der Forschungsfrage. Übereinstimmungen und Widersprüche zur bestehenden Literatur werden herausgearbeitet und begründet.

\subsection{Diskussion von Teilaspekt 1}

Einordnung der Ergebnisse zu Teilaspekt 1 in den Forschungsstand. Mögliche Erklärungen für Übereinstimmungen oder Abweichungen werden diskutiert.

\subsection{Diskussion von Teilaspekt 2}

Einordnung der Ergebnisse zu Teilaspekt 2.


\section{Stärken und Limitationen}

Kritische Reflexion der methodischen Stärken der Arbeit sowie ihrer Grenzen. Mögliche Einschränkungen der Aussagekraft (z.B.\ Stichprobengröße, Studiendesign, Selektionsbias) werden transparent benannt.


\section{Implikationen für die Praxis}

Ableitung von Handlungsempfehlungen für die Praxis auf Basis der Ergebnisse. Empfehlungen werden realistisch und evidenzbasiert formuliert.


\section{Implikationen für die Forschung}

Benennung offener Forschungsfragen, die sich aus den Ergebnissen ergeben, und Empfehlungen für zukünftige Studien.
 \clearpage

% 6. Fazit und Ausblick
% ================================================================
% Kapitel 6: Fazit und Ausblick
% ================================================================

\chapter{Fazit und Ausblick}

Das abschließende Kapitel fasst die Arbeit prägnant zusammen, beantwortet die Forschungsfrage direkt und gibt einen Ausblick auf zukünftige Entwicklungen. Neue Argumente oder Ergebnisse werden hier nicht eingeführt.


\section{Zusammenfassung}

Knappe Zusammenfassung der gesamten Arbeit: Fragestellung, methodisches Vorgehen, zentrale Ergebnisse und deren Diskussion. Der Umfang sollte etwa eine halbe bis eine Seite umfassen.


\section{Beantwortung der Forschungsfrage}

Direkte und präzise Beantwortung der eingangs formulierten Forschungsfrage auf Basis der Ergebnisse und der Diskussion. Die Antwort wird in wenigen Sätzen klar formuliert.


\section{Ausblick}

Benennung offener Fragen, weiterführender Forschungsansätze und perspektivischer Entwicklungen im Themenfeld. Der Ausblick zeigt, welche nächsten Schritte für Wissenschaft und Praxis sinnvoll wären.
 \clearpage

% APA7-Formatierungsbeispiele (nach Fertigstellung entfernen)
% % ================================================================
% APA7-Formatierungsbeispiele
% Diese Datei dient als Referenz und kann nach Fertigstellung
% der Arbeit aus main.tex entfernt werden.
% ================================================================

\chapter{APA7-Formatierungsbeispiele}


% ================================================================
\section{Zitation im Text}
% ================================================================

Ein indirektes Zitat (Paraphrase) wird mit Autor und Jahr angegeben \autocite{Muster2023}. Wenn der Name im Fließtext steht, folgt nur das Jahr in Klammern: Muster et al.\ (2023) beschreiben, dass \ldots

Ein wörtliches Zitat bis 39 Wörter steht in Anführungszeichen im Fließtext: „Hier steht das wörtliche Zitat genau wie im Original" \autocite[S.~45]{Muster2023}.

Ein wörtliches Zitat ab 40 Wörtern wird als eingerücktes Blockzitat dargestellt:

\begin{quote}
Hier steht ein längeres wörtliches Zitat, das mindestens 40 Wörter umfasst. Das Blockzitat wird ohne Anführungszeichen gesetzt und ist links eingerückt. Die Quellenangabe mit Seitenzahl folgt nach dem abschließenden Satzzeichen in Klammern. \autocite[S.~102]{Buch2021}
\end{quote}


% ================================================================
\section{Tabellen nach APA7}
% ================================================================
% Aufbau:
%   \caption{Titel der Tabelle}   ← nur Beschreibungstext, KEIN manuelles "Tabelle X"
%   LaTeX setzt automatisch: "Tabelle 1" (fett) + Zeilenumbruch + Titel (kursiv)
%   Anmerkungen stehen nach der Tabelle, kursiv „Anmerkung."
%   Keine vertikalen Linien – nur \toprule, \midrule, \bottomrule
% ================================================================

\begin{table}[H]
  \caption{Beispielhafte Übersicht mit drei Spalten}
  \label{tab:beispiel}
  \begin{tabularx}{\textwidth}{@{}lXr@{}}
    \toprule
    \textbf{Kategorie} & \textbf{Beschreibung} & \textbf{Wert} \\
    \midrule
    Kategorie A & Kurze Beschreibung der ersten Kategorie  & 1--2 \\
    Kategorie B & Kurze Beschreibung der zweiten Kategorie & 2--3 \\
    Kategorie C & Kurze Beschreibung der dritten Kategorie & nach Bedarf \\
    Kategorie D & Kurze Beschreibung der vierten Kategorie & 3--4 \\
    \bottomrule
  \end{tabularx}
  \par\vspace{2pt}
  {\raggedright\footnotesize\textit{Anmerkung.} Eigene Darstellung auf Basis von \autocite{Buch2021}.\par}
\end{table}


% ================================================================
\section{Abbildungen nach APA7}
% ================================================================
% Aufbau:
%   \caption{Titel der Abbildung}  ← nur Beschreibungstext, KEIN manuelles "Abbildung X"
%   LaTeX setzt automatisch: "Abbildung 1" (fett) + Zeilenumbruch + Titel (kursiv)
%   Beschriftung steht unter der Abbildung
%   Anmerkungen direkt darunter, kursiv „Anmerkung."
% ================================================================

\begin{figure}[H]
  \centering
  % Platzhalter – durch eigene Abbildung ersetzen, z.B.:
  \includegraphics[width=\textwidth]{bilder/beispiel.png}
  \caption{Beispielhafte Darstellung eines Prozessablaufs}
  \label{fig:beispielabbildung}
  \par\vspace{2pt}
  {\raggedright\footnotesize\textit{Anmerkung.} Eigene Darstellung.\par}
\end{figure}


% ================================================================
\section{Überschriftenebenen}
% ================================================================

\subsection{Ebene 2: Linksbündig, Fett}

Text nach einer Ebene-2-Überschrift beginnt in einer neuen Zeile. Kein Einzug in der ersten Zeile nach einer Überschrift ist in APA7 nicht vorgeschrieben – in der Praxis wird er oft weggelassen.

\subsubsection{Ebene 3: Linksbündig, Fett und Kursiv}

Text nach einer Ebene-3-Überschrift beginnt ebenfalls in einer neuen Zeile.

\paragraph{Ebene 4: Eingerückt, Fett, mit Punkt.} Text der Ebene 4 beginnt direkt nach dem Punkt in derselben Zeile und läuft normal weiter.

\subparagraph{Ebene 5: Eingerückt, Fett-Kursiv, mit Punkt.} Text der Ebene 5 beginnt direkt nach dem Punkt in derselben Zeile und läuft normal weiter.
 \clearpage


% ================================================================
% Anhang
% ================================================================
% ================================================================
% Anhang (APA7)
% – Beginnt auf einer neuen Seite
% – Jeder Anhang erhält eine eigene Seite und eine Bezeichnung:
%   Anhang A, Anhang B, …
% – Anhänge werden im Inhaltsverzeichnis aufgeführt
% ================================================================

\appendix
\setcounter{chapter}{0}
\renewcommand{\thechapter}{\Alph{chapter}}

% ---- Anhang A ----
\chapter{Titel Anhang A}
\label{anhang:a}

Hier den Inhalt von Anhang A einfügen (z.B. Erhebungsinstrument, Interviewleitfaden, Rohdaten).


% ---- Anhang B (auskommentiert, bei Bedarf einkommentieren) ----
% \chapter{Titel Anhang B}
% \label{anhang:b}
%
% Hier den Inhalt von Anhang B einfügen.



% ================================================================
% Literaturverzeichnis (APA7 – einheitliche Liste)
% ================================================================
\printbibliography[heading=bibintoc, title={Literaturverzeichnis}]
\clearpage


% ================================================================
% Eigenständigkeitserklärung
% ================================================================
\addchap{Eigenständigkeitserklärung}

Hiermit versichere ich, dass ich die vorliegende Arbeit selbstständig und ohne unzulässige fremde Hilfe verfasst habe. Alle verwendeten Quellen und Hilfsmittel sind vollständig angegeben. Wörtliche und sinngemäße Übernahmen aus anderen Werken sind als solche kenntlich gemacht. Die Arbeit wurde in gleicher oder ähnlicher Form noch keiner anderen Prüfungsbehörde vorgelegt.

\vskip 2cm

\noindent
Ort, Datum \hfill Unterschrift

\vskip 1.5cm

\noindent
\rule{6cm}{0.5pt} \hfill \rule{6cm}{0.5pt}


\end{document}
