% ================================================================
% APA7-konforme Dokumenteinstellungen
% Geeignet für akademische Arbeiten an Hochschulen
% ================================================================

% Dokumentenklasse
% openany: kein Leerseiten-Umbruch vor jedem Kapitel
% chapterprefix=false: kein "Kapitel X" vor dem Titel
\documentclass[
  12pt,
  a4paper,
  oneside,
  openany,
  numbers=noendperiod,
  chapterprefix=false
]{scrbook}


% ---- Seitenränder nach APA7: 2,54 cm (1 Zoll) auf allen Seiten ----
\usepackage[
  top=25.4mm,
  bottom=25.4mm,
  left=25.4mm,
  right=25.4mm
]{geometry}


% ---- Tweaks für KOMA-Script ----
\usepackage{scrhack}


% ---- Schriftart: Times New Roman, 12 pt (APA7-konform) ----
% newtxtext/newtxmath: moderner Ersatz für das veraltete mathptmx,
% liefert hochwertigeres Times New Roman für pdflatex
\usepackage[T1]{fontenc}
\usepackage[utf8]{inputenc}
\usepackage{newtxtext}   % Times New Roman für Fließtext
\usepackage{newtxmath}   % Passende Mathematikschrift
% Alternative bei Verwendung von LuaLaTeX oder XeLaTeX:
%   \usepackage{fontspec}
%   \setmainfont{Times New Roman}


% ---- Sprache: Deutsch ----
\usepackage[ngerman]{babel}


% ---- Mikrotypografie (verbesserte Buchstabenabstände) ----
\usepackage[stretch=10]{microtype}


% ---- Doppelter Zeilenabstand (APA7-Pflicht) ----
\usepackage[doublespacing]{setspace}


% ---- Absatzeinzug nach APA7: 1,27 cm (0,5 Zoll) ----
% Kein zusätzlicher Abstand zwischen Absätzen
\setlength{\parindent}{1.27cm}
\setlength{\parskip}{0pt}


% ---- Captions (Abbildungs- und Tabellenbeschriftungen) ----
\usepackage{caption}


% ---- Grafiken ----
\usepackage{graphicx}


% ---- Positionierung von Abbildungen und Tabellen ----
\usepackage{float}


% ---- Tabellen ----
\usepackage{tabularx}                 % Tabellen mit flexiblen Spaltenbreiten
\usepackage{longtable}                % Seitenübergreifende Tabellen
\usepackage{booktabs}                 % APA7-Tabellenlinien (keine vertikalen Linien)


% ---- Farben ----
\usepackage[table]{xcolor}


% ---- Hyperlinks (ohne farbige Rahmen) ----
\usepackage[hidelinks]{hyperref}


% ---- Literaturverwaltung nach APA7 ----
% Benötigt: biblatex-apa (CTAN-Paket)
\usepackage{csquotes}
\usepackage[
  style=apa,
  backend=biber
]{biblatex}
\DeclareLanguageMapping{ngerman}{ngerman-apa}
\addbibresource{literatur/quellen.bib}


% ---- Kopf- und Fußzeile ----
\usepackage[automark]{scrlayer-scrpage}  % headsepline entfernt: APA7 hat keine Trennlinie
% ================================================================
% Kopf- und Fußzeile
%   – Links:  Aktueller Kapitelname (wechselt automatisch)
%   – Rechts: Seitenzahl
%
% Hinweis APA7:
%   Streng nach APA7 (Manuskript zur Publikation) wäre links ein
%   fester Kurztitel (Running Head). Für Abschlussarbeiten und
%   Hausarbeiten an deutschen Hochschulen ist der Kapitelname
%   jedoch üblicher und prüferseitig erwartet.
%
%   Für festen Kurztitel: \ihead{\small\MakeUppercase{\runninghead}}
%   und \renewcommand{\runninghead}{KURZTITEL} in main.tex setzen.
% ================================================================

\automark{chapter}
\ihead{\small\leftmark}   % Links: aktueller Kapitelname
\chead{}                   % Mitte: leer
\ohead{\thepage}           % Rechts: Seitenzahl

\ifoot*{}
\cfoot*{}
\ofoot*{}

\pagestyle{scrheadings}

