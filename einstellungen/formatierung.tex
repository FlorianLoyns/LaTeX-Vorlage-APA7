% ================================================================
% APA7-Formatierungsanpassungen
% ================================================================


% ---- Serifenschrift für alle Überschriften beibehalten ----
\addtokomafont{disposition}{\rmfamily}


% ================================================================
% APA7 – Keine Nummerierung der Überschriften
% ================================================================
\setcounter{secnumdepth}{-1}


% ================================================================
% APA7 – Fünf Überschriftenebenen
% ================================================================

% Ebene 1 – chapter: Zentriert, Fett, Titelschreibung
% Beispiel:  Einleitung
\setkomafont{chapter}{\normalsize\bfseries}
\renewcommand*{\raggedchapter}{\centering}
\RedeclareSectionCommand[
  beforeskip=0pt,
  afterskip=\baselineskip
]{chapter}

% Ebene 2 – section: Linksbündig, Fett, Titelschreibung
% Beispiel:  Theoretischer Hintergrund
% afterskip muss > 0 sein, sonst behandelt KOMA-Script die Überschrift als Runin
\setkomafont{section}{\normalsize\bfseries}
\RedeclareSectionCommand[
  beforeskip=\baselineskip,
  afterskip=1sp
]{section}

% Ebene 3 – subsection: Linksbündig, Fett und Kursiv, Titelschreibung
% Beispiel:  Theoretische Grundlagen
\setkomafont{subsection}{\normalsize\bfseries\itshape}
\RedeclareSectionCommand[
  beforeskip=\baselineskip,
  afterskip=1sp
]{subsection}

% Ebene 4 – subsubsection: Eingerückt, Fett, Titelschreibung, Punkt, Fließtext folgt
% Beispiel:  Kommunikation im Team.  Text beginnt hier …
\setkomafont{subsubsection}{\normalsize\bfseries}
\RedeclareSectionCommand[
  style=runin,
  beforeskip=0pt,
  afterskip=1em,
  indent=\parindent
]{subsubsection}

% Ebene 5 – paragraph: Eingerückt, Fett und Kursiv, Titelschreibung, Punkt, Fließtext folgt
% Beispiel:  Verbale Kommunikation.  Text beginnt hier …
\setkomafont{paragraph}{\normalsize\bfseries\itshape}
\RedeclareSectionCommand[
  style=runin,
  beforeskip=0pt,
  afterskip=1em,
  indent=\parindent
]{paragraph}


% ================================================================
% APA7 – Tabellen
% APA7-Tabellen haben:
%   • Titel über der Tabelle (Nummer fett, Titel kursiv, neue Zeile)
%   • Keine vertikalen Linien
%   • Horizontale Linien: nur über und unter dem Kopf sowie am Ende
%   • Anmerkungen unter der Tabelle (kursiv „Anmerkung.")
% Empfehlung: Tabellen mit \toprule, \midrule, \bottomrule (booktabs) erstellen
% ================================================================
\setlength{\arrayrulewidth}{0.5pt}


% ================================================================
% APA7 – Blockzitat (quote-Umgebung)
% APA7 §8.27: Blockzitate werden nur linksseitig eingerückt (1,27 cm),
% nicht beidseitig wie LaTeXs Standard-quote.
% ================================================================
\renewenvironment{quote}{%
  \list{}{%
    \leftmargin=1.27cm
    \rightmargin=0pt
    \topsep=0pt
    \parsep=0pt
    \itemsep=0pt
  }%
  \item\relax
}{%
  \endlist
}


% ================================================================
% APA7 – Caption-Format für Abbildungen und Tabellen
% Aufbau: „Abbildung 1" (fett) auf eigener Zeile, Titel (kursiv) darunter
% \DeclareCaptionFormat ist robuster als labelsep=newline in KOMA-Script
% ================================================================
% labelfont=bf macht das Label fett (robuster als \textbf im Format)
% textfont=it macht den Beschreibungstext kursiv
% Das Format {#1#2\\#3} erzeugt: Label + Trenner → Zeilenumbruch → Text
\DeclareCaptionFormat{apa7}{#1#2\\#3}

\captionsetup[figure]{
  format=apa7,
  labelfont=bf,
  textfont=it,
  labelsep=space,
  justification=raggedright,
  singlelinecheck=false
}
\captionsetup[table]{
  format=apa7,
  labelfont=bf,
  textfont=it,
  labelsep=space,
  justification=raggedright,
  singlelinecheck=false,
  position=above
}
