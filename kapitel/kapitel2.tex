% ================================================================
% Kapitel 2: Theoretischer Hintergrund
% ================================================================

\chapter{Theoretischer Hintergrund}

Einführung in den theoretischen Rahmen der Arbeit. Zentrale Konzepte, Modelle und Theorien werden vorgestellt und zueinander in Bezug gesetzt. Der Abschnitt schließt mit der Herleitung der Forschungsfrage.


\section{Begriffe und Definitionen}

Klärung zentraler Begriffe, die für die Arbeit relevant sind. Definitionen werden aus der wissenschaftlichen Literatur abgeleitet und begründet \autocite{Buch2021}.


\section{Forschungsstand}

Überblick über den aktuellen Stand der Forschung zum Thema. Zentrale Studien und Befunde werden systematisch dargestellt und bewertet. Forschungslücken werden identifiziert und begründet.

\subsection{Bisherige Befunde}

Zusammenfassung und kritische Einordnung der relevanten Studien. Übereinstimmungen und Widersprüche in der Literatur werden herausgearbeitet.

\subsection{Forschungslücken}

Benennung offener Fragen, die in der bisherigen Forschung nicht oder unzureichend beantwortet wurden und die Grundlage für die vorliegende Arbeit bilden.


\section{Forschungsfrage und Zielsetzung}

Formulierung der zentralen Forschungsfrage auf Basis der identifizierten Forschungslücke. Die Zielsetzung der Arbeit wird präzise benannt und in den wissenschaftlichen Kontext eingeordnet.
