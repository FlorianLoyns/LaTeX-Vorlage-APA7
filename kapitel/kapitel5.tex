% ================================================================
% Kapitel 5: Diskussion
% ================================================================

\chapter{Diskussion}

Interpretation und kritische Einordnung der Ergebnisse in den theoretischen Rahmen und den Forschungsstand. Die Diskussion beantwortet die Forschungsfrage, setzt die Befunde in Bezug zur Literatur und reflektiert die Stärken und Grenzen der Arbeit \autocite{Schaefer2022}.


\section{Interpretation der Ergebnisse}

Interpretation der zentralen Befunde im Licht der Forschungsfrage. Übereinstimmungen und Widersprüche zur bestehenden Literatur werden herausgearbeitet und begründet.

\subsection{Diskussion von Teilaspekt 1}

Einordnung der Ergebnisse zu Teilaspekt 1 in den Forschungsstand. Mögliche Erklärungen für Übereinstimmungen oder Abweichungen werden diskutiert.

\subsection{Diskussion von Teilaspekt 2}

Einordnung der Ergebnisse zu Teilaspekt 2.


\section{Stärken und Limitationen}

Kritische Reflexion der methodischen Stärken der Arbeit sowie ihrer Grenzen. Mögliche Einschränkungen der Aussagekraft (z.B.\ Stichprobengröße, Studiendesign, Selektionsbias) werden transparent benannt.


\section{Implikationen für die Praxis}

Ableitung von Handlungsempfehlungen für die Praxis auf Basis der Ergebnisse. Empfehlungen werden realistisch und evidenzbasiert formuliert.


\section{Implikationen für die Forschung}

Benennung offener Forschungsfragen, die sich aus den Ergebnissen ergeben, und Empfehlungen für zukünftige Studien.
