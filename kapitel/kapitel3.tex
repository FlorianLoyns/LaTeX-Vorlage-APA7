% ================================================================
% Kapitel 3: Methodik
% ================================================================

\chapter{Methodik}

Überblick über das methodische Vorgehen der Arbeit. Die Wahl der Methode wird begründet und in Bezug auf die Forschungsfrage gerechtfertigt. Das Kapitel folgt dem IMRAD-Schema (Introduction, Methods, Results, and Discussion), das in empirischen wissenschaftlichen Arbeiten weit verbreitet ist.


\section{Forschungsdesign}

Beschreibung und Begründung des gewählten Forschungsdesigns (z.B.\ systematische Literaturrecherche, qualitative Inhaltsanalyse, quantitative Querschnittsstudie). Das Design wird in Bezug auf die Forschungsfrage begründet \autocite{Kapitel2020}.

\subsection{Forschungsparadigma}

Einordnung der Arbeit in das qualitative bzw.\ quantitative Forschungsparadigma und Begründung der Wahl.

\subsection{Gütekriterien}

Darstellung der angewandten Gütekriterien (z.B.\ Reliabilität, Validität, Transferierbarkeit, Glaubwürdigkeit) und Maßnahmen zu ihrer Sicherstellung.


\section{Datenerhebung}

Beschreibung des Vorgehens bei der Datenerhebung. Bei Literaturarbeiten: Suchstrategie, Datenbanken, Ein- und Ausschlusskriterien, PRISMA-Flow. Bei empirischen Arbeiten: Erhebungsinstrument, Feldzugang, Sampling.

\subsection{Suchstrategie und Datenbankrecherche}

Die Literaturrecherche wurde in den Datenbanken \ldots\ durchgeführt. Folgende Suchbegriffe wurden verwendet: \ldots\ Die Suche ergab insgesamt \ldots\ Treffer.

\subsection{Ein- und Ausschlusskriterien}

\begin{table}[H]
  \caption{Ein- und Ausschlusskriterien der Literaturrecherche}
  \label{tab:kriterien}
  \begin{tabularx}{\textwidth}{@{}lXX@{}}
    \toprule
    \textbf{Kriterium} & \textbf{Einschluss} & \textbf{Ausschluss} \\
    \midrule
    Sprache          & Deutsch, Englisch                        & Alle anderen Sprachen        \\
    Erscheinungsjahr & Ab 2014                                  & Vor 2014                     \\
    Studiendesign    & RCT, Kohortenstudie, qualitative Studie  & Fallberichte, Expertenmeinungen \\
    Thema            & Direkt relevant für Forschungsfrage      & Randthemen                   \\
    \bottomrule
  \end{tabularx}
  \par\vspace{2pt}
  {\raggedright\footnotesize\textit{Anmerkung.} Eigene Darstellung.\par}
\end{table}


\section{Datenauswertung}

Beschreibung der Auswertungsmethode (z.B.\ qualitative Inhaltsanalyse nach Mayring, thematische Analyse nach Braun \& Clarke, statistische Auswertung mit SPSS). Vorgehen wird schrittweise erläutert.

\subsection{Auswertungsverfahren}

Begründung der Wahl des Auswertungsverfahrens und Beschreibung der einzelnen Analyseschritte.

\subsection{Kategorienbildung}

Bei qualitativen Arbeiten: Beschreibung der Kategorienbildung (deduktiv, induktiv oder kombiniert) und der Kodierregeln.


\section{Ethische Überlegungen}

Reflexion ethischer Aspekte der Studie (z.B.\ Datenschutz, informierte Einwilligung, Anonymisierung, Genehmigung durch Ethikkommission). Bei Literaturarbeiten: Reflexion möglicher Verzerrungen (Bias).
