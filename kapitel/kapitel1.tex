% ================================================================
% Einleitung
% Überschrift Ebene 1: Zentriert, Fett (APA7)
% ================================================================

\chapter{Einleitung}

Hier den Text der Einleitung einfügen. Die Einleitung führt in das Thema ein, benennt die Problemstellung und formuliert die Forschungsfrage bzw.\ die Zielsetzung der Arbeit. Sie gibt außerdem einen Überblick über den Aufbau der Arbeit.


% ================================================================
% Hinweise zu APA7-Zitierweise (zum Entfernen nach Fertigstellung)
% ================================================================
%
% Kurzzitat im Text:
%   (Nachname, Jahr, S. XX)             → z.B. (Müller, 2021, S. 45)
%   Nachname (Jahr)                     → z.B. Müller (2021)
%   Zwei Autor/innen: (Müller \& Schneider, 2021)
%   Drei und mehr: (Müller et al., 2021)
%
% Indirektes Zitat (Paraphrase):
%   Laut Müller (2021) zeigt sich, dass …
%   (Müller, 2021)
%
% Wörtliches Zitat bis 39 Wörter – im Fließtext in Anführungszeichen:
%   „Wortlaut des Zitats" (Müller, 2021, S. 45)
%
% Wörtliches Zitat ab 40 Wörter – als Blockzitat (eingerückt, ohne Anführungszeichen):
%   \begin{quote}
%     Wortlaut des langen Zitats. (Müller, 2021, S. 45)
%   \end{quote}
%
% Literatureintrag in literatur/quellen.bib einfügen,
% dann im Text mit \autocite{Schlüssel} zitieren.
%
% ================================================================
